%
%  TeX version of the Inform Technical Manual, last revised 18/1/95
%
\newif\iffiles
%
%  Manual macros
%
%  Page layout
%
\newif\ifshutup\shutupfalse
\magnification=\magstep 1
\hoffset=0.15 true in
\voffset=2\baselineskip
%
%  General hacks
%
\def\PAR{\par}
%
%  Font loading
%
\font\medfont=cmr10 scaled \magstep2
\font\bigfont=cmr10 scaled \magstep3
%\def\sectfont{\bf}
\font\sectfont=cmbx12
\def\small{\sevenrm}
\font\rhrm=cmr8
\font\rhit=cmsl8
%
%  Titles
%
\newcount\subsectno    %  Subsection number
\def\rhead{}           %  The running head will go here
%
\def\newsection#1#2{%     To begin a section...
%\global\titletrue%        Declare this as a title page
\xdef\rhead{{\rhrm #1}\quad #2}%       Initialise running head and ssn
\subsectno=0%
\iffiles
\write\conts{\string\sli\string{#1\string}\string{#2\string}\string{\the\pageno\string}}%
\fi
}
%
\def\section#1#2{\vskip 1 true in\goodbreak\newsection{#1}{#2}
\noindent{\sectfont #1\quad #2}\bigskip\noindent}
%
%  Headers and footers
%
\newif\iftitle
\headline={\iftitle\hfil\global\titlefalse%
           \else{\hfil{\rhit \rhead}}%
           \fi}
\footline={\ifnum\pageno<0\hfil{\tenbf\romannumeral -\pageno}%
\else\hfil{\tenbf \number\pageno}\fi}
%\footline={\ifnum\pageno=1\hfil\else\hfil{\tenbf \number\pageno}\fi}
%
%  (Old date-stamping version:)
% \footline={\hfil{\rm \number\pageno}\hfil{\rm \number\day/\number\month}}
%
% If this works I'll be impressed
%

\font\ninerm=cmr9
\font\ninei=cmmi9
\font\ninesy=cmsy9
\font\ninebf=cmbx9
\font\eightbf=cmbx8
\font\ninett=cmtt9
\font\nineit=cmti9
\font\ninesl=cmsl9
\def\ninepoint{\def\rm{\fam0\ninerm}%
  \textfont0=\ninerm
  \textfont1=\ninei
  \textfont2=\ninesy
  \textfont3=\tenex
  \textfont\itfam=\nineit \def\it{\fam\itfam\nineit}%
  \textfont\slfam=\ninesl \def\sl{\fam\slfam\ninesl}%
  \textfont\ttfam=\ninett \def\tt{\fam\ttfam\ninett}%
  \textfont\bffam=\ninebf
  \normalbaselineskip=11pt
  \setbox\strutbox=\hbox{\vrule height8pt depth3pt width0pt}%
  \normalbaselines\rm}

\def\tenpoint{\def\rm{\fam0\tenrm}%
  \textfont0=\tenrm
  \textfont1=\teni
  \textfont2=\tensy
  \textfont3=\tenex
  \textfont\itfam=\tenit \def\it{\fam\itfam\tenit}%
  \textfont\slfam=\tensl \def\sl{\fam\slfam\tensl}%
  \textfont\ttfam=\tentt \def\tt{\fam\ttfam\tentt}%
  \textfont\bffam=\tenbf
  \normalbaselineskip=12pt
  \setbox\strutbox=\hbox{\vrule height8.5pt depth3.5pt width0pt}%
  \normalbaselines\rm}

\parindent=30pt
\def\inpar{\hangindent40pt\hangafter1\qquad}
\def\onpar{\par\hangindent40pt\hangafter0}

\newskip\ttglue
\ttglue=.5em plus.25em minus.15em

\def\orsign{$\mid\mid$}

\outer\def\begindisplay{\obeylines\startdisplay}
{\obeylines\gdef\startdisplay#1
  {\catcode`\^^M=5$$#1\halign\bgroup\indent##\hfil&&\qquad##\hfil\cr}}
\outer\def\enddisplay{\crcr\egroup$$}

\chardef\other=12

\def\ttverbatim{\begingroup \catcode`\\=\other \catcode`\{=\other
  \catcode`\}=\other \catcode`\$=\other \catcode`\&=\other 
  \catcode`\#=\other \catcode`\%=\other \catcode`\~=\other 
  \catcode`\_=\other \catcode`\^=\other
  \obeyspaces \obeylines \tt}
{\obeyspaces\gdef {\ }}

\outer\def\beginstt{$$\let\par=\endgraf \ttverbatim\ninett \parskip=0pt
  \catcode`\|=0 \rightskip=-5pc \ttfinish}
\outer\def\begintt{$$\let\par=\endgraf \ttverbatim \parskip=0pt
  \catcode`\|=0 \rightskip=-5pc \ttfinish}
{\catcode`\|=0 |catcode`|\=\other
  |obeylines
  |gdef|ttfinish#1^^M#2\endtt{#1|vbox{#2}|endgroup$$}}

\catcode`\|=\active
{\obeylines\gdef|{\ttverbatim\spaceskip=\ttglue\let^^M=\ \let|=\endgroup}}

\def\beginlines{\par\begingroup\nobreak\medskip\parindent=0pt
  \nobreak\ninepoint \obeylines \everypar{\strut}}
\def\endlines{\endgroup\medbreak\noindent}

\def\<#1>{\leavevmode\hbox{$\langle$#1\/$\rangle$}}

\def\dbend{{$\triangle$}}
\def\d@nger{\medbreak\begingroup\clubpenalty=10000
  \def\par{\endgraf\endgroup\medbreak} \noindent\hang\hangafter=-2
  \hbox to0pt{\hskip-\hangindent\dbend\hfill}\ninepoint}
\outer\def\danger{\d@nger}
\def\dd@nger{\medskip\begingroup\clubpenalty=10000
  \def\par{\endgraf\endgroup\medbreak} \noindent\hang\hangafter=-2
  \hbox to0pt{\hskip-\hangindent\dbend\kern 1pt\dbend\hfill}\ninepoint}
\outer\def\ddanger{\dd@nger}
\def\enddanger{\endgraf\endsubgroup}

\def\cstok#1{\leavevmode\thinspace\hbox{\vrule\vtop{\vbox{\hrule\kern1pt
       \hbox{\vphantom{\tt/}\thinspace{\tt#1}\thinspace}}
     \kern1pt\hrule}\vrule}\thinspace}

\newcount\exno
\exno=0

\def\xd@nger{%
  \begingroup\def\par{\endgraf\endgroup\medbreak}\ninepoint}

\outer\def\warning{\medbreak
  \noindent\llap{$\bullet$\rm\kern.15em}%
  {\ninebf WARNING}\par\nobreak\noindent}
\outer\def\exercise{\medbreak \global\advance\exno by 1
  \noindent\llap{$\bullet$\rm\kern.15em}%
  {\ninebf EXERCISE \bf\the\exno}\par\nobreak\noindent}
\def\dexercise#1{\global\advance\exno by 1
  \medbreak\noindent\llap{$\bullet$\rm\kern.15em}%
  #1{\eightbf ~EXERCISE \bf\the\exno}\hfil\break}
\outer\def\dangerexercise{\xd@nger \dexercise{\dbend}}
\outer\def\ddangerexercise{\xd@nger \dexercise{\dbend\dbend}}


\newwrite\ans%
\newwrite\conts%
\iffiles
\immediate\openout\conts=$.games.infocom.ftp.toolkit.mandir.conts
\fi

\iffiles\else\outer\def\answer#1{\par\medbreak}\shutuptrue\fi

\newwrite\inx
\ifshutup\else
\immediate\openout\inx=$.games.infocom.ftp.toolkit.mandir.inxdata
\fi
\def\marginstyle{\sevenrm %
  \vrule height6pt depth2pt width0pt } %

\newif\ifsilent
\def\specialhat{\ifmmode\def\next{^}\else\let\next=\beginxref\fi\next}
\def\beginxref{\futurelet\next\beginxrefswitch}
\def\beginxrefswitch{\ifx\next\specialhat\let\next=\silentxref
  \else\silentfalse\let\next=\xref\fi \next}
\catcode`\^=\active \let ^=\specialhat
\def\silentxref^{\silenttrue\xref}

\newif\ifproofmode
\proofmodetrue %

\def\xref{\futurelet\next\xrefswitch}
\def\xrefswitch{\begingroup\ifx\next|\aftergroup\vxref
  \else\ifx\next\<\aftergroup\anglexref
    \else\aftergroup\normalxref \fi\fi \endgroup}
\def\vxref|{\catcode`\\=\active \futurelet\next\vxrefswitch}
\def\vxrefswitch#1|{\catcode`\\=0
  \ifx\next\empty\def\xreftype{2}%
    \def\next{{\tt\text}}%
  \else\def\xreftype{1}\def\next{{\tt\text}}\fi %
  \edef\text{#1}\makexref}
{\catcode`\|=0 \catcode`\\=\active |gdef\{}}
\def\anglexref\<#1>{\def\xreftype{3}\def\text{#1}%
  \def\next{\<\text>}\makexref} %
\def\normalxref#1{\def\xreftype{0}\def\text{#1}\let\next=\text\makexref}

\def\makexref{\ifproofmode%
  \xdef\writeit{\write\inx{\text\space!\xreftype\space
    \noexpand\number\pageno.}}\iffiles\writeit\fi
  \else\ifhmode\kern0pt \fi\fi
 \ifsilent\ignorespaces\else\next\fi}

\newdimen\fullhsize
\def\fullline{\hbox to\fullhsize}
\let\lr=L \newbox\leftcolumn

\def\doubleformat{\shipout\vbox{\makeheadline
    \fullline{\box\leftcolumn\hfil\columnbox}
    \makefootline}
  \advancepageno}
\def\tripleformat{\shipout\vbox{\makeheadline
    \fullline{\box\leftcolumn\hfil\box\midcolumn\hfil\columnbox}
    \makefootline}
  \advancepageno}
\def\columnbox{\leftline{\pagebody}}

\newbox\leftcolumn
\newbox\midcolumn
\def\beginindex{
\fullhsize=6.5true in \hsize=2.1true in
 \global\def\makeheadline{\vbox to 0pt{\vskip-22.5pt
      \fullline{\vbox to8.5pt{}\the\headline}\vss}\nointerlineskip}
 \global\def\makefootline{\baselineskip=24pt \fullline{\the\footline}}
 \output={\if L\lr
   \global\setbox\leftcolumn=\columnbox \global\let\lr=M
 \else\if M\lr
   \global\setbox\midcolumn=\columnbox \global\let\lr=R
 \else\tripleformat \global\let\lr=L\fi\fi
 \ifnum\outputpenalty>-20000 \else\dosupereject\fi}
\begingroup
  \parindent=1em \maxdepth=\maxdimen
  \def\par{\endgraf \futurelet\next\inxentry}
  \obeylines \everypar={\hangindent 2\parindent}
  \exhyphenpenalty=10000 \raggedright}
\def\inxentry{\ifx\next\sub \let\next=\subentry
  \else\ifx\next\endindex \let\next=\vfill
  \else\let\next=\mainentry \fi\fi\next}
\def\endindex{\mark{}\break\endgroup
\supereject
\if L\lr \else\null\vfill\eject\fi
\if L\lr \else\null\vfill\eject\fi
}
\let\sub=\indent \newtoks\maintoks \newtoks\subtoks
\def\mainentry#1,{\mark{}\noindent
  \maintoks={#1}\mark{\the\maintoks}#1,}
\def\subentry\sub#1,{\mark{\the\maintoks}\indent
  \subtoks={#1}\mark{\the\maintoks\sub\the\subtoks}#1,}

\def\subsection#1{\medbreak\par\noindent{\bf #1}\qquad}

%  For contents

\def\cl#1#2{\bigskip\par\noindent{\bf #1}\quad {\bf #2}}
\def\li#1#2#3{\smallskip\par\noindent\hbox to 5 in{{\bf #1}\quad #2\dotfill #3}}
\def\sli#1#2#3{\par\noindent\hbox to 5 in{\qquad\item{#1}\quad #2\dotfill #3}}
\def\fcl#1#2{\bigskip\par\noindent\hbox to 5 in{\phantom{\bf 1}\quad {\bf #1}\dotfill #2}}

% Epigrams

\def\poem{\begingroup\narrower\narrower\narrower\obeylines\ninepoint}
\def\widepoem{\begingroup\narrower\narrower\obeylines\ninepoint}
\def\verywidepoem{\begingroup\narrower\obeylines\ninepoint}
\def\quote{\medskip\begingroup\narrower\narrower\noindent\ninepoint}
\def\poemby#1#2{\par\smallskip\qquad\qquad\qquad\qquad\qquad -- #1, {\it #2}
\tenpoint\endgroup\bigskip}
\def\widepoemby#1#2{\par\smallskip\qquad\qquad\qquad -- #1, {\it #2}
\tenpoint\endgroup\bigskip}
\def\quoteby#1{\par\smallskip\qquad\qquad\qquad\qquad\qquad
 -- #1\tenpoint\endgroup\bigskip}
\def\endquote{\par\tenpoint\endgroup\medskip}

%
%  End of macros
\def\subtitle#1{\bigbreak\noindent{\bf #1}\medskip}

\titletrue\tenpoint

\centerline{\bigfont The Inform Technical Manual}
\vskip 0.5in
\centerline{\sl for revision 5.4, last updated 18/1/95}
\vskip 0.5in 

\sli{1}{Introduction}{2}
\sli{2}{Recondite directives}{2}
\sli{3}{Unusual constant forms}{4}
\sli{4}{String indirection and low strings}{5}
\sli{5}{Game control commands and keyboard reading}{5}
\sli{6}{Obselete commands}{7}
\sli{7}{The abbreviations optimiser}{8}
\sli{8}{Dictionary and parsing table formats}{9}
\sli{9}{Porter's notes}{12}
\sli{10}{Geography and history of the source code}{14}
\vfill\eject

\section{1}{Introduction}

This is a short collection of notes on low-level matters covering what
is neither in {\sl The Inform Designer's Manual} nor the assembly-language
documentation in {\sl The Specification of the Z-machine}.

The Designer's Manual is, however, intended to be entirely self-contained
for all practical purposes.  If this document contains nothing either
interesting or useful, I feel I shall have achieved my purpose.

It contains much of the commentary which used to be in the source code's
header, such as its modification history, notes on porting the Inform
compiler to new machines and documentation of obselete or internally-used
features.  I anticipate revising this (though not necessarily the
Designer's Manual) each time the source code is updated.

\medskip
\hbox to\hsize{\hfill\it Graham Nelson}
\hbox to\hsize{\hfill\it Magdalen College, Oxford}
\hbox to\hsize{\hfill\it September 1994}
\medskip

This is now updated to v1405 (still of release 5.4) and covers recent
maintenance.
\medskip
\hbox to\hsize{\hfill\it January 1995}
\ninepoint

\section{2}{Recondite directives}

These are the directives airily dismissed as `recondite' in \S A1 to the
Designer's Manual.
\beginstt
Default <cname> <value>;
\endtt
If the constant has not yet been defined, define it with this value.
(In |Verblib| this is used to give constants like |MAX_CARRIED| their
default values if the main game source has not already set them;
hence the name.)
\beginstt
Stub <rname> <n>;
\endtt
If the routine has not yet been defined, define one which has $n$ local
variables and simply returns false.  (Setting the number of local variables
prevents the game from calling a routine with more arguments than it has
local variables to put them in; this should not do any harm to the
interpreter, but neither does a little caution.)  This is how ``entry
point'' routines are handled: the |Grammar| library file stubs out any
undeclared entry points.
\beginstt
Dictionary <name> <text>;
\endtt
Enters |<text>| in dictionary, and makes a new constant for its address.
This is not so much recondite as obselete; nowadays one would write
something like
\beginstt
Constant frog_word 'frog';
\endtt
but in any case now that one can write simply |'frog'| the need has gone
away.
\beginstt
System_file;
\endtt
Declares the present file to be a `system file'.  The only way in which
these differ from other files is that if Inform has been told to |Replace|
a given routine, it will ignore a definition of this routine in a `system
file'.  Thus |Parser| and |Verblib| are system files, and conceivably
other user-written library extensions (for magic, say) might want to be.
\beginstt
Lowstring <name> <string>;
\endtt
Puts |string| in the ``low strings" area of the Z-machine (an area in the
lowest 64K of memory which holds static strings, usually to hold
abbreviations), and creates a constant with the given |name| to hold its
word address.  Any string which is to be used with the |@| string escape
must be declared in this low strings area.  (But the use of the |@|
string escape is clumsy and there are probably better ways to get the
effect in Inform 5.)
\beginstt
Version <v>;
\endtt
sets the game file version (3 for Standard games, 5 for Advanced; 4 and 6
are present for completeness).  This directive isn't so much recondite
as redundant; the preferred way is to either set |-v3| or some such at
the command line, or to include a |switches| directive, e.g.
\beginstt
Switches v3;
\endtt
\medskip
The remaining directives are for debugging Inform only:
\beginstt
Listsymbols;
Listdict;
Listverbs;
Listobjects;
\endtt
are fairly self-explanatory (be warned: they can produce a lot of output).
In addition, a number of tracing modes can be turned on and off in
mid-pass:
\beginstt
Trace    Btrace    Ltrace    Etrace
NoTrace  NoBtrace  NoLtrace  NoEtrace
\endtt
|Trace| is an assembly-language style trace, with addresses and bytes
as compiled; |Btrace| is the same, but produced on both passes, not just
on pass 2; |Ltrace| traces each internal line of code; and |Etrace|, the
highest-level of these, traces the expression evaluator at work by
printing out the expression trees made and the assembly source these are
reduced to.  (A more vehement, less legible version is |etrace full|,
which shows the process in minute detail.)

\section{3}{Unusual constant forms}

There are more constant forms in Inform 5 than are dreamt of in the
Designer's Manual.  Some are obselete, others obscure.  To begin with,
Inform predefines a number of constants which are used by the library:
\beginstt
  adjectives_table   (byte address)
  preactions_table   (byte address)
  actions_table      (byte address)
  code_offset        (packed address of code)
  strings_offset     (packed address of strings)
  version_number     (3 or 5 as appropriate)
  largest_object     (the number of the largest created object + 255)
  dict_par1
  dict_par2
  dict_par3      
\endtt
which can be read by something like
\beginstt
lookup = #adjectives_table;
\endtt
One does occasionally want to know the largest object number in
high-level code, but the library provides a variable |top_object|
such that the legal object numbers are
$$ 1 \leq n \leq \hbox{|top_object|} $$
and using this is preferable.

The |dict_par| constants are byte offsets into a dictionary entry
of the three bytes of data about the word, and are provided because
these offsets are different between Standard and Advanced games;
thus, the parser uses these constants to ensure portability between
the two.
\medskip

A constant beginning |#a$| means ``the action number of this action
routine''.  Thus, |#a$TakeSub| is equivalent to the more usual
|##Take|.
\medskip

A constant beginning |#w$|, followed by a word of text, has as value the
address of the given word in the dictionary (Inform will give an
error at compile time if no such word is present).  Largely obselete.
\medskip

A constant beginning |#n$|, followed by a word of text, has as value the
address of the given word in the dictionary (Inform adds it to the
dictionary as a new word if it is not already there).  Thus,
|#n$leopard| is equivalent to |'leopard'|.  However, this constant form
is still useful to enter single-letter words into the dictionary
(like |y|, which the parser defines as an abbreviation for ``yes'')
since |'y'| would instead mean the ASCII value of the character `y'.
\medskip

A constant beginning |#r$|, followed by a routine name, gives the (packed)
address of the given routine.  This is chiefly useful for changing the
routine-valued properties of an object in mid-game, e.g.
\beginstt
lamp.before = #r$NewBeforeRoutine;
\endtt
where |NewBeforeRoutine| is defined as a global routine somewhere.

\section{4}{String indirection and low strings}

Inside a static string (in double-quotes), the string escape |@nn|,
an |@| sign followed by a two digit number, means ``print the
$n$-th string variable here''.  |nn| is a decimal number from |00|
to |31|.  Now such a variable string can be set with the
\beginstt
String <number> <low-string-constant>;
\endtt
which means that any string to be used in this way has to have been
defined as a ``low string'' (see above).  For example,
\beginstt
Lowstring L_Frog "little green frog";
...
String 0 #L_Frog;
"You notice a @00!^";
\endtt
will result in the output
\beginstt
You notice a little green frog!
\endtt
Actually, since the first 32 entries of the ``synonyms table'' in the
Z-machine are reserved for this purpose, the command |String n x|
is in fact equivalent to
\beginstt
(0-->12)-->n=x;
\endtt
Due to a minor design infelicity of the Z-machine, the more
friendly-looking usage
\beginstt
String 0 "illegal frog";
\endtt
will work in a Standard game but may unpredictably fail in an
Advanced one exceeding 128K in length; hence the need to ensure
all relevant strings are ``low'' (in the bottom 128K of memory).

\section{5}{Game control commands and keyboard reading}

\beginstt
quit;
\endtt
(Actually an assembly language opcode.)  This quits the game
(at once, with no confirmatory question to the user): all games must
end this way, since it is illegal to return from the |Main| routine).
\beginstt
restart;
\endtt
(Similarly an opcode.)  Restarts the game exactly to its initial state,
losing the previous state for good.
\beginstt
save <label>;
restore <label>;
verify <label>;
\endtt
Tries to save or load in a saved game file, or to verify that the
existing story file is not corrupted (by calculating a checksum and
comparing it against the one in the header).  In each case, |jump|
to the given label if successful (otherwise run on into the next
statement as usual).  |save| and |restore| are actually commands and
not opcodes because the relevant opcodes function differently between
Standard and Advanced games; this command ensures portability.

\bigskip
\beginstt
Read <a> <b> [<routine>];
\endtt
This reads from the keyboard (printing no prompt: it is assumed this
has already been done) into buffer |a| and tokenises it into buffer |b|.
(|a| and |b| are expected to point to global string variables, defined
by something like
\beginstt
Global a string 120;
\endtt
meaning that |a->0| contains the number 120, and that |a->1| to
|a->120| are bytes of available read/write memory.)  In Standard games,
this command automatically redisplays the status line.  In Advanced ones,
if no routine is given then Inform compiles code to emulate the
Standard game status line automatically; if a routine is given, this
is called instead, and is expected to update the status line itself.
See the Designer's Manual for an example of such a routine.

\noindent After |read| has taken place:
\item{$\bullet$} |a->1| holds the number of characters typed;
\item{$\bullet$} the text, unterminated, is held in |a->2| to |a->(a->1 + 1)|;
\item{$\bullet$} |b->1| holds the number of words typed (note that commas
and full stops become separate words in their own right);
\item{$\bullet$} from byte 2 onward, |b| contains 4-byte blocks, one for
each word, in the form
\itemitem{0,1} byte address of dictionary entry if word is known, 0 otherwise;
\itemitem{2} number of letters in word
\itemitem{3} first character of word in the |a| buffer.

\noindent More flexible tokenising and keyboard-reading methods are available
by resorting to assembly language; see the |aread| opcode and the
`special effects' section of the Designer's Manual.

\section{6}{Obselete commands}

Inform 5 continues to provide a number of out-dated features from Inform 1
to 4; `out-dated' in the sense that there are now much better ways to do
the same things.  The old features have not been removed because the
largest Inform program in existence (`Curses') still makes use of them;
their further use is not encouraged.
\medskip
The |put| command takes the form:
\beginstt
put <addr> byte <index> <v>;
put <addr> word <index> <v>;
\endtt
which are the old way to use arrays, now superceded by
\beginstt
addr->index=v;
addr-->index=v;
\endtt
\medskip
The |write| command can be used to write to many properties of an object
at once:
\beginstt
write <object> <p1> <v1> [<p2> <v2>...];
\endtt
and was useful in the days when the only alternative was using the
|@put_prop| assembly opcode, but is now superceded by lines like
\beginstt
lamp.time_left = 0;
\endtt
which are clearer and more consistent.
\medskip
Before Inform provided C-style |for| loops, it had BASIC-style ones:
these were the so-called `old-style |for| loops',
\beginstt
for <var> <start> to <finish> { ...code... }
\endtt
which were restricted in having only simple finish values (i.e., not
compound expressions) and in requiring braces around the code (even
if it contained only a single statement).  The effect can be duplicated
with
\beginstt
for (<var>=<start> : <var> <= <finish> : <var>++) ...code...
\endtt
one form of a much more general and flexible construct.

\section{7}{The abbreviations optimiser}

When the game becomes full, 8 to 10\% of its length can be saved by making
use of text abbreviations: a method under which up to 64 commonly occurring
phrases can be abbreviated whenever they occur.  This makes no difference
to text as seen by the player.  Because checking for these causes a speed
overhead (again, of about 10\%) and it isn't worthwhile unless a game is
very large, Inform does not do so except in economy mode (compiling with
the switch |-e| on).  Abbreviations must be declared explicitly, before
any other text appears, by a directive such as:
\beginstt
Abbreviate "the ";
\endtt
This causes "the " to be stored internally as only 2 text chunks (5-bit
segments), rather than 4, whenever it occurs: which is very often.
Only 64 may be declared (the remaining 32 slots in the Z-machine's
``synonyms table'' being kept for string indirection).

To see how good your current choice of abbreviations is, try compiling with
the |-f| (frequencies) option set, which will count the number of times each
abbreviation is used, and work out how many bytes it saved.  For instance,
|" the "| occurs some 2445 times in `Curses'.  Experiment soon reveals that
parts of speech and words like |"there"| make big savings, but that almost
any proper noun makes little difference.

Infocom's own compiler does not seem to have chosen abbreviations very
rigorously, since Infocom story files contain just such a naive list.  (This
may have been wise from the point of view of printing speed in the days of
much slower computers.)

In any case, the |-u| option of Inform (if your computer is large enough and
fast enough to make this feasible) will try to work out a nearly-optimal set
of abbreviations.

The algorithm for doing so is too complex to give here: see the source code.
Briefly, it runs in two phases: building a table of cross-references, and
then running a number of passes looking for good substrings and choosing
good antichains from the partially ordered set resulting.  (The main problem
being that abbreviations interfere with each other: taking both of
|"the"| and |"the "| will not give the same saving as the individual savings
added up.)  The result is not guaranteed to be optimal but seems pretty good.
The output it finally produces is a list of legal Inform |Abbreviate|
commands which can be pasted into source code.

Since there are something like
$$ 2^{300000} $$
possible choices for a typical-sized game, this is bound to be an
expensive job.  A 128K game takes about 45 seconds to compile on my machine,
and slightly under two hours to optimise.  There are three passes, of which
the first is by far the longest.

Reasonable guesswork and experiment (resulting in the words suggested in
earlier editions of this manual) actually doesn't perform too badly, but
when I first optimised a 128K version of `Curses', the |-u| option saved
1200 bytes over the best choices made by hand: here is the selection
produced, in the form of |-f| output:
\beginlines
|    How frequently abbreviations were used, and roughly how many|
|    bytes they saved:  ('_' denotes spaces)|
|       you   668/  444         with   144/  190        which    92/  182   |
|       urs    58/   38         tion   142/  188          ter   274/  182   |
|       t_w   134/   88          t_s   117/   77          t_o   164/  108   |
|       t_i   167/  110          ing   960/  639         ight   187/  248   |
|       her   283/  188          e_w   146/   96          e_s   160/  106   |
|       e_o   227/  150          e_i   245/  162          e_a   254/  168   |
|       der    87/   57          d_s    61/   40          d_o   122/   80   |
|       d_i    82/   54          d_a   122/   80          and   560/  372   |
|       all   289/  192          You   297/  394         This    47/   92   |
|       The   384/  510      Meldrew    28/  108        It_is    40/  104   |
| Aunt_Jemima  15/  102           ._   680/  452           ,_  1444/  962   |
|      's_~    42/  109        's_no    41/  106          _un   105/   69   |
|       _to   708/  471        _the_  1328/ 2654          _th   578/  384   |
|       _ro   110/   72          _pr    95/   62          _po    78/   51   |
|       _no   246/  163          _ma   165/  109          _lo   119/   78   |
|       _ho    87/   57          _hi    99/   65          _ha   309/  205   |
|       _gr    67/   44          _ga    60/   39        _from    94/  186   |
|      _for   185/  245          _fi   130/   86          _fa    97/   64   |
|       _ex    89/   58          _ea    61/   40        _door    46/   90   |
|       _di   110/   72         _con    88/  116         _com    72/   94   |
|       _cl    81/   53         _can   164/  217          _ba   120/   79   |
|       _a_   587/  390   |
\endlines
On a version of `Curses' taking up about 240K, using abbreviations saved
about 23000 bytes and added 9 seconds to a 91-second compilation time.

It's interesting how few words in common the naive and optimised lists
have.  Only two proper nouns survived, and they provide the only longish
words.  |"is "| as such turned out not to be worthwhile.  |" the "| was
perhaps obvious in retrospect, but I didn't think of it.  The best strategy
for abbreviating seems to be to choose three-character strings which make
a fractional saving each (only one Z-character each time, for the most part)
but which occur very often indeed.

Note also that another 32 abbreviations (which could be accommodated, if the
string indirections mechanism were dropped) would be little help, as
the least worthwhile of these already saves only 38 bytes or so.

\section{8}{Dictionary and parsing table formats}

Some of the tables Inform writes into the Z-machine have formats which are
not imposed by the Z-machine specification but by Inform's own conventions,
and these are covered here.  These conventions are based on (but different
to) those used in the middle-period Infocom games.
\medskip

{\ninebf Adjectives} are numbered downwards from |$ff| in order of their appearance in
defined grammar.  The adjective table contains 4-byte entries:
\beginstt
<dictionary address of word>  00  <adjective number>
----2 bytes-----------------  ----2 bytes----------- 
\endtt
To make life more interesting, these entries are stored in reverse
order (i.e., lowest adjective number first).  The address of this table is
rather difficult to deduce from the file header information, so the constant
|#adjectives_table| is set up by Inform to refer to it.
\medskip

The {\ninebf grammar table} address is stored in word 7 (i.e. bytes 14 and 15)
of the header.  The table consists of a list of two-byte addresses to
the entries for each word.  This list is immediately followed by these
entries, one after another.

An entry consists of one byte giving the number of lines and then that
many 8-byte lines.  These lines have the form
\beginstt
<objects>  <sequence of words>  <action number>
--1 byte-  ----6 bytes--------  --1 byte-------
\endtt
|<objects>| is the number of objects which need to be supplied: eg, 0 for
``inventory", 1 for ``take frog", 2 for ``tie rope to dog".  The sequence
of words gives up to 6 tokens following the verb, to be matched in order.
The token values are given by the table:
\beginstt
   noun             0  
   held             1  
   multi            2  
   multiheld        3  
   multiexcept      4  
   multiinside      5  
   creature         6  
   special          7  
   number           8
   (noun=Routine)   16+parsing-routine-number
   (Routine)        48+parsing-routine-number
   (scope=Routine)  80+parsing-routine-number
   (attribute)      128+attribute number
   (adjective)      adjective number
   ...reserved...   9-15, 112-127
\endtt
Parsing routines have addresses which are too large to store in a single
byte.  Instead they are numbered from 0, and their (packed) addresses are
stored in the {\ninebf preactions table} of the story file.  (This is called
``preactions table" because of what the original Infocom parser used it
for; the Inform library parser has no such concept as `preaction'.)

The sequence is padded out to 6 bytes with zeros.  (This is a tiresome
convention, as it means that the value 0 can only be understood by
looking back at what has come before, but it's too late to change it now.)
\medskip

{\ninebf Actions} are numbered from 0 upwards in order of appearance in the
grammar.  (Whereas fake actions are numbered from |$ff| down, but that's
another story.)  The packed addresses of the corresponding action routines
are stored in the {\ninebf actions table}.  Once again, Inform puts this table
in its conventional place, but its address is difficult to work out and
so the constant |#actions_table| is set up to hold it.
\medskip

{\ninebf Verbs} are numbered from |$ff| downwards in order of appearance,
with synonyms getting the same number (thus, ``get'' and ``take'' have
the same verb number); they are entered into the dictionary as they are
defined in grammar.
\medskip

In the {\ninebf dictionary} header, Inform defines only three characters as
`separators' which break up words in tokenisation: these are full stop,
comma and open-double-quote.  (In theory the Z-machine allows any list
here, but these three are conventional in old Infocom story files.)

Inform writes dictionary entries consisting of the word itself, plus
three data bytes.  (This makes them 7 bytes long in Standard games,
9 in Advanced.)
The entries are in alphabetical order, and look like:
\beginstt
<the text of the word>  <flags>  <verb number>  <adjective number>
----4 or 6 bytes------  --1 b--  ----1 byte---  ----1 byte--------
\endtt
The text is stored in the usual text format, thus allowing up to 6 or 9
characters.  These data bytes can be safely accessed (portably between
either format of game) by, e.g.
\beginstt
address->#dict_par1
\endtt
which reads the flags byte of the word at |address|.

The flags (chosen once again to conform loosely to Infocom conventions, not
for any sensible reason) have the eight bits
\beginstt
7      6  5  4  3     2      1      0
<noun> .. .. .. <adj> <spec> <meta> <verb>
\endtt
|<verb>|, |<noun>| and |<adj>| mean the word can be a verb, noun or adjective;
the |<spec>| bit means the word was inserted by a |Dictionary| command in the
program, except that |<verb>| words also have the |<spec>| bit set (ours not to
wonder why).

Verbs declared as |"meta"| have the |<meta>| bit set.  (These are such
out-of-world experiences as ``save'' and ``score''.)

Note that a word can be any combination of these at once.  It can even be
simultaneously a verb, adjective and noun, and will be understood as such
in different contexts.

\vfill\eject
\section{9}{Porter's notes}

The following ports have (generally) successfully been made:
\beginstt
  the Commodore Amiga under SAS/C              Christopher A. Wichura
  the Acorn Archimedes under Norcroft C        (the author)
  the Atari ST                                 Charles Briscoe-Smith
  Linux under gcc (essentially as per Unix)    Spedge, aka Dark Mage
  the Apple Macintosh                          Robert Pelak
  the Mac, under the Programmer's Workshop     Brad Jones
  OS/2 32-bit mode under IBM's C Set++         John W. Kennedy
  386+ IBM PCs, eg. Microsoft Visual C/C++     Toby Nelson
  small IBM PCs under QuickC                   Bob Newell
  Unix under gcc (or big IBM PCs under djgpp)  Dilip Sequeira
  VAX mainframes under Digital's VAX C         (the author)
\endtt
(Apologies to anyone left out.)  Recent ports have been relative painless,
and on many machines (particularly those with 32-bit integers and flat
memory maps) the code has simply compiled and worked first-time without
trouble.  Executables from most of the above ports may be found
ready-compiled in the archive |ftp.gmd.de|.

Porters are asked to name such executables, when they post them, with
a filename which clearly indicates the machine and the revision number
(e.g. by ending |.5.4|); and to email the author, if possible, with
details of any modifications they needed to make, so that the main
source can be improved.  In particular, |diff|s (that is, differences
between the source code as last posted, and the source as compiled
for the machine in question) would be a great help.

See the |header.h| file for some make-files for different compilers.

The code assumes that |long int| is at least 32 bits long, though
plain |int| can be either 16 or 32.

The general procedure is as follows: all code special to your port should
appear inside |#IFDEF|s, with a constant for your port being defined
at the head of the file.  A block of definitions should appear in the
header file along with the others.  For instance, here is the block for
Unix:
\beginlines
|#ifdef UNIX|
|#define MACHINE_STRING   "Unix"|
|#define Source_Prefix    ""|
|#define Source_Extension ".inf"|
|#define Include_Extension ".h"|
|#define Code_Prefix      ""|
|#define Code_Extension   ".z3"|
|#define V5Code_Extension ".z5"|
|#define Transcript_Name "game.txt"|
|#define Debugging_Name  "game.dbg"|
|extern char Temp1_Name[], Temp2_Name[];|
|#define Temp1_Hdr "/tmp/InformTemp1"|
|#define Temp2_Hdr "/tmp/InformTemp2"|
|#define DEFAULT_MEMORY_SIZE LARGE_SIZE|
|#define US_POINTERS|
|#endif|
\endlines
Notice that since some abysmally poor C compilers (such as VAX C, which
has the affrontery to call itself ANSI) require that all |#| directives
begin on the first column of the source code, you should abide by this
as well.

|MACHINE_STRING| should name the machine.  The prefixes and extensions
are defaults for filenaming conventions.  Under Unix, then, ``frog''
will mean |frog.inf| it it's a source file, |frog.h| if an include
file (such as a library file), |frog.z3| or |frog.z5| it it's to be
written as a Standard or Advanced game respectively.  The
|Transcript_Name| and |Debugging_Name|s hold names for the transcribed
text and debugging information files (both optionally produced).
Arrangements for naming temporary files (used for temporary storage
space during compilation) are more vexed on some machines; under Unix,
they are given unique names depending on the current process, for
instance.  Code to work out these names is defined in |files.c|,
conditionally compiled when |UNIX| is defined.  If there is no such
multi-tasking issue, or you can't be bothered, you can just write
something like
\beginlines
|#define Temp1_Name "Inftmp1.tmp"|
|#define Temp2_Name "Inftmp2.tmp"|
\endlines

Set the |DEFAULT_MEMORY_SIZE| to |LARGE_SIZE| if you can; but if this is
too consumptive for a small model of your machine, choose |SMALL_SIZE|.
(In any case this default can be over-ridden on the command line.)

There are a few options:

|TEMPORARY_FILES| means `use temporary files for scratch workspace'.
The alternative is to use a good deal of extra RAM, say 256K or so,
and possibly to have trouble allocating it since it will need to
have large contiguous chunks.

|US_POINTERS| uses unsigned rather than signed |char *| pointers when
calculating things like checksums; you may need this if your compiler
signs |char| by default.

|TIME_UNAVAILABLE| indicates that the ANSI what's-the-date-today routines
can't be used.  (The serial number of a game will then be 940000 unless
otherwise set by a |Serial| directive in it.)

|PROMPT_INPUT| indicates that the usual ANSI command-line arguments
system cannot be used, which makes Inform ask questions instead.  This
is tiresome, so if you can find an alternative, take it.

|GRAHAM| indicates that you are Graham Nelson (not to be recommended).

\medskip
All claims of memory are routed through |my_malloc| and |my_calloc|
in |files.c|.  The Quick C port shows what can be done in the case when
your compiler will not allow you to allocate more than 64K contiguously
without great trouble.

\medskip
If two copies of Inform on different machines are given
{\sl identical} source code (and have identical version numbers, and
identical ideas about what today's date is) then they should produce
{\sl identical} game files.  If your port can pass this test for the
`Advent' example game, it's probably in good shape.  A quicker test
is to try typing ``verify'' into a game produced by your port; often
accidents of porting are shown up by wrong checksums in this way.

Don't worry if you can't get some of the more unusual switches to work,
such as the gargantuan |-u|, but please cope with the user asking for
it by giving some suitable error message.

\section{10}{Geography and history of the source code}

The Inform source code is written in ever stricter and more pragmatic ANSI
C.  It consists of a header file |header.h| and eight files of code:
\beginstt
asm.c  express.c  files.c  inform.c  inputs.c  tables.c  zcode.c  symbols.c
\endtt
(see the attached map).
\midinsert
\hrule\smallskip
\centerline{\sl A tourist's map of the Inform archipelago}
\medskip
$$\vbox{%
\settabs\+\indent&\qquad&\qquad&\qquad&reallyreallyquitealongsampleword&\cr%
\+ Main (top level) &&&&& |inform.c|\cr
\+ & Initialisation\cr
\+ & Command line switches\cr
\+ & Top level line parser\cr
\+ & & Compiler\cr
\smallskip
\+ & & & Assignments and conditions && |express.c|\cr
\+ & & & Expression evaluator\cr
\smallskip
\+ & & Assembler directives &&& |asm.c|\cr
\+ & & Line assembler\cr
\+ & & & Constant evaluator\cr
\+ & & & Make attributes/properties\cr
\smallskip
\+ & & & Make objects and classes && |tables.c|\cr
\+ & & & Make globals\cr
\+ & & & Make verbs\cr
\+ & & & Make dictionary\cr
\+ & & & Make actions\cr
\+ & & & Print diagnostics\cr
\+ & Construct output file\cr
\smallskip
\+ & & & Text translation && |zcode.c|\cr
\+ & & & Z-code database\cr
\+ & & & Reserved words table\cr
\+ & Abbreviations optimiser\cr
\smallskip
\+ & & & Symbols table maker && |symbols.c|\cr
\smallskip
\+ & & & Preprocessor stack && |inputs.c|\cr
\+ & & & Character-level parsing\cr
\+ & & & Error reporting\cr
\smallskip
\+ & & & File handling && |files.c|\cr
\+ & & & Fatal errors\cr
\+ & & & Debugging information file\cr
\+ & & & Memory management\cr
\smallskip
\+ & & & & Comments & |header.h|\cr
\+ & & & & |#define|s\cr
\+ & & & & Integer types\cr
\+ & & & & Structures\cr
\+ & & & & Extern declarations\cr}$$
\smallskip\hrule
\endinsert
Inform runs in two passes, like an assembler.  Mostly it does the same
things on pass 2 as pass 1 (but is able to sort out forward references);
but there are exceptions (the dictionary is insertion-sorted and hashed
on pass 1, then strung together in the right order on pass 2, for instance).

It tokenises one line at a time (and does not make elaborate parse trees,
which is why it is not good at hanging elses).  Lines are divided
up between directives, assembler opcodes and statements.  Statements
are normally converted back into sequences of assembler lines, which
are held on the `preprocessor stack' to be processed next (before
the next statement from the source).  Some complex statements even
compile to simpler ones and so on down.  In this way the original
source becomes a stream of assembly language.

Objects and classes are stored in a compressed format similar to their
final format, to save on memory.

Only one error is allowed through per original source statement (which
prevents assembly-language errors caused by poor error recovery in
some cases).

The modification history of Inform 1 to 5 is as follows.  Note that some of
the earlier remarks are archaic and out-of-date, representing features now
superceded.  Apologies to those whose corrections went in without their
names being recorded!
\smallskip
\hrule
\medskip
\noindent{\bf Inform 1}
\beginlines
|     The first archive release (0.5) was on April 30th 1993.               |
\endlines
\medskip
\noindent{\bf Inform 2}
\beginlines
|     The second archive release (0.6) had the following improvements:      |
|                                                                           |
|     One #ifdef ARCHIMEDES altered to correct a bug in non-Archimedes      |
|     version                                                               |
|     Checking on the MAX_ACTIONS limit put in ("Curses" exceeded 100!)     |
|     Checking on MAX_STATIC_STRINGS put in; -m information extended        |
|     -x (hash printing) option introduced                                  |
|     -a (list assembly lines only) option, and ATRACE/NOATRACE introduced  |
|     Void prototypes explicitly declared (void)                            |
|     Defunct Inform directives "STRING" and "SET" removed                  |
|     Opcode data now made static, and faster opcode-parsing routine put in |
|     Preprocessor stack rewritten, and now checking for overflow           |
|     Showdict produces more useful output                                  |
|     Filename extension #defines added                                     |
|     Command line parsing improved                                         |
|     Some ASCII assumptions removed                                        |
|     Typedefs added to force integers to be 32-bits long                   |
|     Memory management heavily reformed, at the expense of a certain charm |
|                                                                           |
|     USE_TEMPORARY_FILES version: if this is #defined, scratch files       |
|       amounting to at most about 100K and 50K respectively are used to    |
|       hold the code and static strings areas; this saves about another    |
|       150K.                                                               |
|       (At worst three files are simultaneously open under this regime.)   |
|       The temporary file names are #define'd below.  They are             |
|         automatically deleted.                                            |
\endlines
\medskip
\noindent{\bf Inform 3}
\beginlines
|   The third release (1.0) is generally tidied and reorganised: most       |
|   of the sillier variable and routine names have been made more           |
|   comprehensible.                                                         |
|   It is also 3 to 6 times faster; thanks due to Dilip Sequeira for        |
|   profiling output, and also David Moore for his... comments.             |
|       (November 1993)                                                     |
|                                                                           |
|   Program improvements in the third release:                              |
|                                                                           |
|     @xx string indirection via the synonyms table added                   |
|     Objects allowed to have multiple internal names                       |
|     New constant form #n$word... added                                    |
|     And #r$routine...                                                     |
|     New high-level commands "write" and "give" for easier object amending |
|     Fatal errors fractionally more informative                            |
|     Non-fatal errors quite a lot more informative, and better worded      |
|     Grievous bug in stack long slot routines fixed                        |
|     The checksum and length words are now properly set (though few        |
|       interpreters need them)                                             |
|     Error checking on exceeding MAX_VERBS                                 |
|     -e (economy mode) added: causes abbreviations to be worked out,       |
|       slowly (this is why it is only an option)                           |
|     #SWITCHES directive added                                             |
|     -i (ignore switches) and -o (print offsets) added                     |
|     Checking added on whether routines have too many local variables (the |
|       Z-machine crashes in a very strange way if so!)                     |
|     Minor bug in printing object tree fixed                               |
|     Two unused bytes spare at end of property defaults table are now      |
|       zeroed                                                              |
|     Temporary files now deleted after use                                 |
|     Checking on excessively long variable names added                     |
|     STATUSLINE directive added (for games with hours/minutes on the       |
|       status line)                                                        |
|     The former SMALL_MEMORY compilation option is now mandatory.          |
|       (Previously, Inform could be compiled so that it read source files  |
|       into an enormous buffer, rather than reading them twice through a   |
|       bit at a time.  This could only be useful on machines with huge     |
|       memory and very slow filing systems, of which there are few, and it |
|       complicated the code.)                                              |
|     The way input file names are processed has been reformed: they are    |
|       now not altered if they contain a '.' or a '/'                      |
|     INCLUDE directive added, so that Inform #includes files like C        |
|     Old -p (both passes) directive renamed -b, and new -p (percentage     |
|       breakdown)                                                          |
|     Warnings added: variables not used; checking that Main behaves        |
|       properly; small bug in line counting fixed; checking on number of   |
|       function arguments                                                  |
|     Meta-verbs added                                                      |
|     -f (frequencies) and -t (assembly trace) switches added               |
|     Small bug to do with stubbed routines removed                         |
|     Possibly unused bytes (due to word alignment) in data, now zeroed     |
|       (so that different machines will not produce different game files)  |
|     -f now calculates bytes yielded by abbreviations                      |
|     New SERIAL directive for machines without access to today's date      |
|     Now handles more complicated multiple expressions within the same     |
|       command                                                             |
|     New STRING command added for writing to the synonyms table            |
|     New FONT command for proportional fonts control                       |
|     New DEFAULT and STUB directives, for stubbing undeclared CONSTANTs    |
|       and code                                                            |
|     Checking on no. of attributes and properties added, and               |
|       property-counting                                                   |
|                                                                           |
|   Speed improvements in the third release:                                |
|                                                                           |
|     The following have been rewritten in the interests of speed and       |
|       not being O(n^2) for the sake of it: the line reader and tokeniser, |
|       management of local variables, the dictionary builder, the text     |
|       translator, the line parser and the symbols table (courtesy of hash |
|       coding by Dilip).                                                   |
|                         Curses    Dejavu       (compiling times (seconds) |
|                                                 on my machine)            |
|        Release 2...        300        45       (including 1-2 seconds for |
|        Tokeniser & locals  205        26        printing statistics)      |
|        Dictionary           89        19                                  |
|        Symbols hashing      74        17                                  |
|        Tokeniser II         69        16                                  |
|        Abbreviations        55        16                                  |
|        Hashing reserveds    49        14                                  |
|                                                                           |
|   Compatibility improvements in the third release:                        |
|                                                                           |
|     The sort_number routine has been rewritten at the suggestion of Jon   |
|       Drukman in order to defend against compilers determined to sign     |
|       chars; and so have some structure definitions and variable types    |
|     Subtraction of pointers is now done by an easily altered macro (the   |
|       point being that you can't always subtract by casting to int, if    |
|       int is 16 bit or if you have a dire MSDOS-like memory map)          |
|     File naming improved slightly                                         |
|     The two points where ASCII is used now go through translate_to_ascii  |
|     Some stupid alterations made for VAX C compatibility                  |
|       (in the idiot world of VAX C, # commands must start on column 1,    |
|       x=-1 is read as x-=1, typedef isn't ANSI, the word "signed" is      |
|       rejected, values like MAX_INT are wrongly set and string consts     |
|       don't concatenate)                                                  |
|     A general rewrite has been made to sort out 16-bit from 32-bit        |
|       integers: Inform now properly works when int is 16 bit by default.  |
|     VAX version now working (so presumably Inform does not rely on the    |
|       order of bytes in a word)                                           |
|     Long constants explicitly declared so (to keep Borland C++ happy)     |
|     Because some C compilers (especially PC ones) don't like large static |
|       arrays there's now an ALLOCATE_BIG_ARRAYS option (#define PC forces |
|       it) which uses calloc to allocate memory from the heap for them.    |
|     Altogether Inform is going to need about 200K of workspace, and       |
|       that's that: in a big flat memory machine, this will split about    |
|       equally between static arrays and dynamic allocation.  With         |
|       ALLOCATE_BIG_ARRAYS set it will be almost entirely dynamically      |
|       allocated.                                                          |
|     If PROMPT_INPUTS is defined (and the VAX and PC versions force this), |
|       Inform gets file names and options by prompting for keyboard input, |
|       rather than using a Unix-style command line.                        |
|     If TIME_UNAVAILABLE is defined, Inform doesn't try to use strftime    |
|       and doesn't enter today's date for the serial number: the           |
|       programmer will have to use a SERIAL directive in Inform, instead.  |
|                                                                           |
|     Improvements made for Release 3a:  (Dec 7th 1993)                     |
|                                                                           |
|      The AMIGA port option added (following Christopher Wichura)          |
|      #define US_POINTERS option added                                     |
|      A few constants (eg. MAX_BANK_SIZE) slightly increased, as "Curses"  |
|        needed it when very close indeed to maximum possible size          |
|     A few void routines which weren't explicitly called (void) now are    |
|     The use of local text buffers by routines has been reformed, so that  |
|       although there's now about 6K more of array allocation, the stack   |
|       needed during runs of Inform is very much smaller (previously       |
|       machines with less than 32K stack couldn't manage).                 |
|     The tokeniser now recognises tab characters (outside string literals) |
|       as spaces (Inform previously gave errors when it found these).      |
|     The begin_pass routines are now more legible                          |
\endlines
\medskip
\noindent{\bf Inform 4}
\beginlines
|     Miscellaneous improvements made for Release 4:  (January 20th 1994)   |
|                                                                           |
|     Checking on file I/O errors (previously Inform only checked errors    |
|     which occurred on opening files, so never noticed disc space          |
|     running out)                                                          |
|     Lamentable wrong-verify-code bug in R3a (caused by misplaced #endif)  |
|       fixed, and checksum calculation rewritten in a truly paranoid way   |
|       for better portability to machines signing char's                   |
|     "p[syns]=0x80" made to work when char's are signed (-128 to 127); a   |
|       few redundant initialisations of variables removed                  |
|     Minor tracing bug in R3a (only) fixed                                 |
|     Source code reformatted to 79 columns wide for troglodyte monitors    |
|     New typedef of "zip" (for char / unsigned char) to simplify the       |
|       the US_POINTERS option                                              |
|     Heavy reorganisation and division into seven separately-compiled      |
|       files; variables sorted into extern and static throughout           |
|     Optimisation on void-context function calls (saving about 300 bytes   |
|       on a v-3 file of size 128K!)                                        |
|     Conditional compilation added: #IFDEF, #IFNDEF, #IFNOT, #ENDIF        |
|     Slow and memory-intensive abbreviations optimiser added: -u switch    |
|     Text transcription (-r) added                                         |
|     Property and attribute "alias" introduced (using code suggested by    |
|       Art Dyer)                                                           |
|     Properties and attributes formally separated as types, and a warning  |
|       introduced for the common accident of missing out a comma in a      |
|       property list                                                       |
|     Warning put in for over-long property data (in Version 3 files)       |
|     Property operators ".", ".&", ".#" added                              |
|     Direct array and property assignments (eg., a->2=3;) added            |
|     Expression evaluator tracing improved (and "etrace full" added)       |
|     Expressions generally reformed, and complicated conditions added      |
|     Braces made optional for simple if clauses                            |
|     Negative constants now tokenised correctly and allowed                |
|     Unary minus, ++ and -- added; x+-1 optimised to x-1                   |
|     "Children" and other in-lined object functions added                  |
|     Recondite bug in expressions with nested function calls fixed         |
|     Preprocessor stack fully rewritten in a cleverer way (and it worked   |
|       first time!)                                                        |
|     New-style "for", "objectloop" added (they didn't, though); bug in the |
|       old "do...until" code fixed                                         |
|     Assignments can now take the form of a comma-separated-list           |
|     Braces made optional for arbitrary (nested) new-style constructs      |
|     Fixed miscellaneous bugs and finally rewrote the expression evaluator |
|       in a tokenised way, losing about 50 calls to strcmp per operator -  |
|       which made no noticeable difference to run time but I feel better   |
|     my_malloc and my_calloc given the correct int type - size_t           |
|     Rare bug with large constant initial values for global variables when |
|       int is 2-byte, fixed                                                |
|     Microsoft Visual C/C++ port added (following Toby Nelson)             |
|     Line tracing format made more legible (at long last)                  |
|     strcmp no longer used with possibly null strings (which is allowed    |
|       by ANSI to crash the machine, and does under Unix)                  |
|                                                                           |
|   New Version-5 features in Release 4:                                    |
|                                                                           |
|     -v3 and -v5 switches, and VERSION directive, added to switch between  |
|       producing version 3 (Standard) and version 5 (Advanced) games       |
|     Rewritten statistics routines; changes to some array limits           |
|     New Advanced opcodes added                                            |
|     Optimisation of calls in v-5 code to make use of variant opcodes      |
|     STYLE command for bold-face, underlining, reverse video               |
|     READ command replaces the old v-3 opcode of the same name, and        |
|       emulates it (with an optional status-line-routine) in version 5     |
|     Dictionary routines rewritten for either 6- or 9-character accuracy   |
|     New constants #dict_par1, -2, -3 and #version_number                  |
|     SAVE, RESTORE commands replace the old v-3 opcodes, so as to emulate  |
|       them in v-5                                                         |
|     #IFV3 and #IFV5 directives added                                      |
|     Different file extensions/prefixes for version 5 files                |
|     BOX command added                                                     |
|                                                                           |
\endlines
\medskip
\noindent{\bf Inform 5}
\beginlines
|     Many minor improvements and bug fixes; object classes; inheritance;   |
|     embedded routines in object definitions; cosmetic improvements        |
|                                                                           |
|     June 1st 1994; revised June 12th (5.1), and again June 19th (5.2)     |
|                                                                           |
|     A few corrections by Christopher Wichura for the AMIGA option,        |
|       and Amiga makefile commented above - 25/1                           |
|     Variable name in grow_branch() changed on advice of David Ingram      |
|       (it was called opcode, which was also a typedef'd name) - 14/2      |
|     Check on exceeding MAX_ROUTINES put in (finally!) - 26/2              |
|     Minor bug to do with property data exceeding 10 bytes fixed - 13/3    |
|     Assembler slightly rewritten, and the new opcode naming system        |
|       introduced: a few minor changes, numerous (unuseful) additions - "  |
|     Versions 4 and 6 (-v4, -v6) added for completeness - 16/3             |
|     Some minor tidying up of code suggested by Bob Newell, and the        |
|       USE_TEMPORARY_FILES option finally made to work on PCs - 17/3       |
|     Bob's Quick C port added - 17/3                                       |
|     Nasty (but extremely unlikely) bug to do with data area fixed, and    |
|       memory allocation for this made more flexible - 30/3                |
|     Really miserable, vile bug in the expression evaluator to do with     |
|       exactly the case  function(a-1,b)  fixed - 14/4                     |
|     Errors added for duplicated and misplaced "else"s - rather important  |
|       since Inform handles hanging elses slightly naively - 15/4          |
|     Charles Briscoe-Smith's Atari ST port added - 20/4                    |
|     John Kennedy's OS/2 port added                                        |
|     Robin Watts' Archimedes throwback code added                          |
|     Bob Newell's fix for the } brace underflow bug added (about time too) |
|     Robert Pelak's Macintosh port added - 26/4                            |
|     Testing on size of quoted strings added (previously Inform could      |
|       crash if they were more than 2K long) - 27/4                        |
|     -j (list objects as made) switch added - 27/4                         |
|     Better reporting of output file opening errors - 27/4                 |
|     Code generator fixed to produce only safe calls to get_prop_len       |
|       (calls with non-existent properties crash some interpreters) - 1/5  |
|     Code for drawing quotation boxes made smaller and better - 1/5        |
|     Updated the gcc makefile on Dilip's advice - 2/5                      |
|     Put error check in for misplaced "switches" directive, after such a   |
|       mistake confused me for ages - 17/5                                 |
|     Embedded routines in object declarations added - 18/5                 |
|     ##Action form added - 18/5                                            |
|     Forward references to constants now understood (a thorny problem      |
|       because of the long/short storage dilemma) - 18/5                   |
|     Classes and inheritance, embedded routines, fake actions added        |
|     Dictionary routines fixed to allow e.g. "y2" and "pipe-dream" - 19/5  |
|     -n (print property/attribute numbers as allocated) added              |
|     <Action ...> added; dictionary address constants reformed - 20/5      |
|     "Nearby" declarations, and object locations defaulted to "nothing"    |
|     Bare strings understood as print_ret                                  |
|     <<Action...>> added - 21/5                                            |
|     Additive properties; "name" made additive; more logical punctuation   |
|       of object definitions - 26/5                                        |
|                                                                           |
|     At this point the code sat in the incoming directory at ftp.gmd.de    |
|       for a while and was discovered and looted, after which the          |
|       following improvements were made, producing Inform 5.2:             |
|                                                                           |
|     Old "inform.c" file divided about equally into two, making new file   |
|       "express.c" for the expression evaluator - a long overdue change    |
|     Minor bug (to do with 32 properties being exceeded in V5) fixed       |
|     Memory allocation tracing (-m) and freeing improved                   |
|     The old ALLOCATE_BIG_ARRAYS option is now mandatory                   |
|     Memory $ commands (on the command line) added: memory consumption     |
|       reduced: symbols table memory allocation reformed                   |
|     Two minor, convoluted bugs in line counting for error reports fixed   |
|     Temporary files not left lying around after errors have occurred      |
|     Grammar table extensions added                                        |
|                                                                           |
|     Revision 5.3 (as appeared in Acorn User magazine's cover disc)        |
|                                                                           |
|     Bob Newell's Quick C port revised, and his compatibility improvements |
|       incorporated (with the result that some ints are now int32s, and    |
|       array typing is more careful with double indirection: this explains |
|       why several apparent char *'s come from int32 ** structures)        |
|     The #inclusions of the header now written in more orthodox C form     |
|     Bug in reporting the missing-comma-after-] error, noticed by          |
|       Teo Kwang Liak, fixed                                               |
|     Header for Amiga version corrected by Christopher Wichura             |
|                                                                           |
|     Revision 5.4                                                          |
|                                                                           |
|     Archimedes temporary-file prefixing added                             |
|     Three memory settings moved int -> int32, one cast in symbols fixed,  |
|       the huge printf divided into two, some %d's made %ld (at suggestion |
|       of Robert Pelak and Bob Newell)                                     |
|     Tab characters ignored when quoted strings split across lines         |
|       (this overlooked case pointed out by Andrew McMurry)                |
|     tracing_mode now set to 0 (as it should have been before)             |
|     @@decimal-number syntax added to text translation, so that untypeable |
|       chars (e.g. accented letters, graphic chars) can be put in text     |
|     Error added for spurious code after #include line                     |
|     Opcode names standardised by agreement with Mark Howell; with         |
|       the very last transuranic opcode added                              |
|     -n (property number) tracing improved, in the course of...            |
|     ...fixing an unfortunate bug in additivity of property inheritance    |
|       which resulted in some copies of Advent crashing                    |
|     The * syntax for "-g debug on this routine only" added                |
|     Really foolish bug in tokeniser (first token buffer 1 byte too short  |
|       with lines beginning with long-named function calls) fixed          |
|     Extensive support for the Infix debugger added                        |
|     String literals longer than 509 chars, the ANSI safe limit, divided   |
|       (at request of Brad Jones)                                          |
|     Constant form 'word' now allowed in object/class property defns       |
|     'scope=' token support added, and 'name=' consequently renumbered     |
|     Bug in input routines which didn't allow '"' fixed                    |
|     Error messages generally revised and clarified                        |
|     Bug to do with crashing on a long initial token which isn't a string  |
|       removed, and error messages protected against excessive length      |
|                                                                           |
|     Slight updates in releasing 5.4 (Oct 1994):                           |
|     Minor variable definition mishap fixed                                |
|     More ANSI printing of hex addresses which overflow 16-bit ints        |
|     Bug to do with default property values on 16-bit-int machines fixed   |
|     The Sibelius Bug fixed (this caused some if statements not to put     |
|       braces properly around single statements, when the first character  |
|       of the corresponding source line was a TAB).  (This bug is so known |
|       because Bob Newell had a sudden flash of inspiration and fixed it   |
|       on his laptop while killing time in a hotel in Minot, North Dakota, |
|       where he was staying to record a radio programme of Scandinavian    |
|       music, including Sibelius' Concerto for Violin and Orchestra in D   |
|       minor.  The opening cadenza is reminiscent of tab stops...)         |
|     More of Bob's int32/int corrections made                              |
|                                                                           |
|     Maintenance, Dec 94 and Jan 95:                                       |
|     The empty string "" now causes an error (rather than going wrong)     |
|     Extend "first" grammar comes out in the right order now, and doesn't  |
|       waste a grammar line (as it used to)                                |
|     Inheritance of attributes numbered 32 and over from classes used to   |
|       have regrettable side-effects (unexpected extra attributes sometimes|
|       appeared), but hopefully no longer                                  |
|     Expression evaluator mended.  Arithmetic like a-b-c now implicitly    |
|       brackets as (a-b)-c, not a-(b-c).  This is obviously preferable.    |
|       Similarly a->b->c now means (a->b)->c, which I think is the right   |
|       thing to do but just might be a dangerous change for code already   |
|       in existence: well, I'll take the risk                              |
|     Boring expression tracing bug fixed                                   |
|     MAX_ROUTINES checking was out by one: now mended                      |
|     Opcode same_parent now renamed jin (since it turns out to jump if in) |
|       and the "in" condition correspondingly more efficiently coded to    |
|       use this opcode, saving compilation time and code space (slightly). |
|     Games with large quantities of extra grammar could crash Inform in    |
|       v1404 and before, owing to a verb name table overflowing without    |
|       being checked.  This table space is expanded (as it's cheap on      |
|       memory anyway), and its size is a new memory setting, MAX_VERBSPACE.|
|     The display of memory settings has been alphabetically sorted         |
\endlines
\hrule
\smallskip
\end
